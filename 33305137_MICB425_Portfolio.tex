\documentclass[]{article}
\usepackage{lmodern}
\usepackage{amssymb,amsmath}
\usepackage{ifxetex,ifluatex}
\usepackage{fixltx2e} % provides \textsubscript
\ifnum 0\ifxetex 1\fi\ifluatex 1\fi=0 % if pdftex
  \usepackage[T1]{fontenc}
  \usepackage[utf8]{inputenc}
\else % if luatex or xelatex
  \ifxetex
    \usepackage{mathspec}
  \else
    \usepackage{fontspec}
  \fi
  \defaultfontfeatures{Ligatures=TeX,Scale=MatchLowercase}
\fi
% use upquote if available, for straight quotes in verbatim environments
\IfFileExists{upquote.sty}{\usepackage{upquote}}{}
% use microtype if available
\IfFileExists{microtype.sty}{%
\usepackage{microtype}
\UseMicrotypeSet[protrusion]{basicmath} % disable protrusion for tt fonts
}{}
\usepackage[margin=1in]{geometry}
\usepackage{hyperref}
\hypersetup{unicode=true,
            pdftitle={Pretty\_html},
            pdfauthor={Leilynaz Malekafzali (33305137)},
            pdfborder={0 0 0},
            breaklinks=true}
\urlstyle{same}  % don't use monospace font for urls
\usepackage{color}
\usepackage{fancyvrb}
\newcommand{\VerbBar}{|}
\newcommand{\VERB}{\Verb[commandchars=\\\{\}]}
\DefineVerbatimEnvironment{Highlighting}{Verbatim}{commandchars=\\\{\}}
% Add ',fontsize=\small' for more characters per line
\usepackage{framed}
\definecolor{shadecolor}{RGB}{248,248,248}
\newenvironment{Shaded}{\begin{snugshade}}{\end{snugshade}}
\newcommand{\KeywordTok}[1]{\textcolor[rgb]{0.13,0.29,0.53}{\textbf{#1}}}
\newcommand{\DataTypeTok}[1]{\textcolor[rgb]{0.13,0.29,0.53}{#1}}
\newcommand{\DecValTok}[1]{\textcolor[rgb]{0.00,0.00,0.81}{#1}}
\newcommand{\BaseNTok}[1]{\textcolor[rgb]{0.00,0.00,0.81}{#1}}
\newcommand{\FloatTok}[1]{\textcolor[rgb]{0.00,0.00,0.81}{#1}}
\newcommand{\ConstantTok}[1]{\textcolor[rgb]{0.00,0.00,0.00}{#1}}
\newcommand{\CharTok}[1]{\textcolor[rgb]{0.31,0.60,0.02}{#1}}
\newcommand{\SpecialCharTok}[1]{\textcolor[rgb]{0.00,0.00,0.00}{#1}}
\newcommand{\StringTok}[1]{\textcolor[rgb]{0.31,0.60,0.02}{#1}}
\newcommand{\VerbatimStringTok}[1]{\textcolor[rgb]{0.31,0.60,0.02}{#1}}
\newcommand{\SpecialStringTok}[1]{\textcolor[rgb]{0.31,0.60,0.02}{#1}}
\newcommand{\ImportTok}[1]{#1}
\newcommand{\CommentTok}[1]{\textcolor[rgb]{0.56,0.35,0.01}{\textit{#1}}}
\newcommand{\DocumentationTok}[1]{\textcolor[rgb]{0.56,0.35,0.01}{\textbf{\textit{#1}}}}
\newcommand{\AnnotationTok}[1]{\textcolor[rgb]{0.56,0.35,0.01}{\textbf{\textit{#1}}}}
\newcommand{\CommentVarTok}[1]{\textcolor[rgb]{0.56,0.35,0.01}{\textbf{\textit{#1}}}}
\newcommand{\OtherTok}[1]{\textcolor[rgb]{0.56,0.35,0.01}{#1}}
\newcommand{\FunctionTok}[1]{\textcolor[rgb]{0.00,0.00,0.00}{#1}}
\newcommand{\VariableTok}[1]{\textcolor[rgb]{0.00,0.00,0.00}{#1}}
\newcommand{\ControlFlowTok}[1]{\textcolor[rgb]{0.13,0.29,0.53}{\textbf{#1}}}
\newcommand{\OperatorTok}[1]{\textcolor[rgb]{0.81,0.36,0.00}{\textbf{#1}}}
\newcommand{\BuiltInTok}[1]{#1}
\newcommand{\ExtensionTok}[1]{#1}
\newcommand{\PreprocessorTok}[1]{\textcolor[rgb]{0.56,0.35,0.01}{\textit{#1}}}
\newcommand{\AttributeTok}[1]{\textcolor[rgb]{0.77,0.63,0.00}{#1}}
\newcommand{\RegionMarkerTok}[1]{#1}
\newcommand{\InformationTok}[1]{\textcolor[rgb]{0.56,0.35,0.01}{\textbf{\textit{#1}}}}
\newcommand{\WarningTok}[1]{\textcolor[rgb]{0.56,0.35,0.01}{\textbf{\textit{#1}}}}
\newcommand{\AlertTok}[1]{\textcolor[rgb]{0.94,0.16,0.16}{#1}}
\newcommand{\ErrorTok}[1]{\textcolor[rgb]{0.64,0.00,0.00}{\textbf{#1}}}
\newcommand{\NormalTok}[1]{#1}
\usepackage{longtable,booktabs}
\usepackage{graphicx,grffile}
\makeatletter
\def\maxwidth{\ifdim\Gin@nat@width>\linewidth\linewidth\else\Gin@nat@width\fi}
\def\maxheight{\ifdim\Gin@nat@height>\textheight\textheight\else\Gin@nat@height\fi}
\makeatother
% Scale images if necessary, so that they will not overflow the page
% margins by default, and it is still possible to overwrite the defaults
% using explicit options in \includegraphics[width, height, ...]{}
\setkeys{Gin}{width=\maxwidth,height=\maxheight,keepaspectratio}
\usepackage[normalem]{ulem}
% avoid problems with \sout in headers with hyperref:
\pdfstringdefDisableCommands{\renewcommand{\sout}{}}
\IfFileExists{parskip.sty}{%
\usepackage{parskip}
}{% else
\setlength{\parindent}{0pt}
\setlength{\parskip}{6pt plus 2pt minus 1pt}
}
\setlength{\emergencystretch}{3em}  % prevent overfull lines
\providecommand{\tightlist}{%
  \setlength{\itemsep}{0pt}\setlength{\parskip}{0pt}}
\setcounter{secnumdepth}{0}
% Redefines (sub)paragraphs to behave more like sections
\ifx\paragraph\undefined\else
\let\oldparagraph\paragraph
\renewcommand{\paragraph}[1]{\oldparagraph{#1}\mbox{}}
\fi
\ifx\subparagraph\undefined\else
\let\oldsubparagraph\subparagraph
\renewcommand{\subparagraph}[1]{\oldsubparagraph{#1}\mbox{}}
\fi

%%% Use protect on footnotes to avoid problems with footnotes in titles
\let\rmarkdownfootnote\footnote%
\def\footnote{\protect\rmarkdownfootnote}

%%% Change title format to be more compact
\usepackage{titling}

% Create subtitle command for use in maketitle
\newcommand{\subtitle}[1]{
  \posttitle{
    \begin{center}\large#1\end{center}
    }
}

\setlength{\droptitle}{-2em}
  \title{Pretty\_html}
  \pretitle{\vspace{\droptitle}\centering\huge}
  \posttitle{\par}
  \author{Leilynaz Malekafzali (33305137)}
  \preauthor{\centering\large\emph}
  \postauthor{\par}
  \predate{\centering\large\emph}
  \postdate{\par}
  \date{version March 31, 2018}

\usepackage{booktabs}
\usepackage{longtable}
\usepackage{array}
\usepackage{multirow}
\usepackage[table]{xcolor}
\usepackage{wrapfig}
\usepackage{float}
\usepackage{colortbl}
\usepackage{pdflscape}
\usepackage{tabu}
\usepackage{threeparttable}
\usepackage[normalem]{ulem}

\begin{document}
\maketitle

{
\setcounter{tocdepth}{2}
\tableofcontents
}
\section{R Markdown PDF Challenge}\label{r-markdown-pdf-challenge}

The following assignment is an exercise for the reproduction of this
.html document using the RStudio and RMarkdown tools we've shown you in
class. Hopefully by the end of this, you won't feel at all the way this
poor PhD student does. We're here to help, and when it comes to R, the
internet is a really valuable resource. This open-source program has all
kinds of tutorials online.

\begin{figure}
\centering
\includegraphics{/Users/Leilynaz/documents/image.gif}
\caption{\url{http://phdcomics.com/} Comic posted 1-17-2018}
\end{figure}

\subsection{Challenge Goals}\label{challenge-goals}

The goal of this R Markdown html challenge is to give you an opportunity
to play with a bunch of different RMarkdown formatting. Consider it a
chance to flex your RMarkdown muscles. Your goal is to write your own
RMarkdown that rebuilds this html document as close to the original as
possible. So, yes, this means you get to copy my irreverant tone exactly
in your own Markdowns. It's a little window into my psyche. Enjoy =)

\textbf{hint: go to the \href{http://phdcomics.com/}{PhD Comics website}
to see if you can find the image above}\\
\emph{If you can't find that exact image, just find a comparable image
from the PhD Comics website and include it in your markdown}

\subsubsection{Here's a header!}\label{heres-a-header}

Let's be honest, this header is a little arbitrary. But show me that you
can reproduce headers with different levels please. This is a level 3
header, for your reference (you can most easily tell this from the table
of contents).

\paragraph{Another header, now with
maths}\label{another-header-now-with-maths}

Perhaps you're already really confused by the whole markdown thing.
Maybe you're so confused that you've forgotton how to add. Never fear!
\sout{A calculator} R is here:

\begin{Shaded}
\begin{Highlighting}[]
\DecValTok{1231521}\OperatorTok{+}\DecValTok{12341556280987}
\end{Highlighting}
\end{Shaded}

\begin{verbatim}
## [1] 1.234156e+13
\end{verbatim}

\subsubsection{Table Time}\label{table-time}

Or maybe, after you've added those numbers, you feel like it's about
time for a table!\\
I'm going to leave all the guts of the coding here so you can see how
libraries (R packages) are loaded into R (more on that later). It's not
terribly pretty, but it hints at how R works and how you will use it in
the future. The summary function used below is a nice data exploration
function that you may use in the \textsuperscript{future}.

\begin{Shaded}
\begin{Highlighting}[]
\KeywordTok{library}\NormalTok{(knitr)}
\KeywordTok{kable}\NormalTok{(}\KeywordTok{summary}\NormalTok{(cars),}\DataTypeTok{caption=}\StringTok{"I made this table with kable in the knitr package library"}\NormalTok{)}
\end{Highlighting}
\end{Shaded}

\begin{longtable}[]{@{}lcc@{}}
\caption{I made this table with kable in the knitr package
library}\tabularnewline
\toprule
& speed & dist\tabularnewline
\midrule
\endfirsthead
\toprule
& speed & dist\tabularnewline
\midrule
\endhead
& Min. : 4.0 & Min. : 2.00\tabularnewline
& 1st Qu.:12.0 & 1st Qu.: 26.00\tabularnewline
& Median :15.0 & Median : 36.00\tabularnewline
& Mean :15.4 & Mean : 42.98\tabularnewline
& 3rd Qu.:19.0 & 3rd Qu.: 56.00\tabularnewline
& Max. :25.0 & Max. :120.00\tabularnewline
\bottomrule
\end{longtable}

And now you've almost finished your first RMarkdown! Feeling excited? We
are! In fact, we're so excited that maybe we need a big finale eh?
Here's ours! Include a fun gif of your choice!

\begin{figure}
\centering
\includegraphics{https://media.giphy.com/media/rmi45iyhIPuRG/giphy.gif}
\caption{}
\end{figure}

\section{Module 1:}\label{module-1}

\subsection{\texorpdfstring{Evidence Worksheet for ``Prokaryotes: The
Unseen
Majority''}{Evidence Worksheet for Prokaryotes: The Unseen Majority}}\label{evidence-worksheet-for-prokaryotes-the-unseen-majority}

\href{https://www.ncbi.nlm.nih.gov/pmc/articles/PMC33863/}{Whitman et al
1998}

\subsubsection{Learning Objectives}\label{learning-objectives}

Describe the numerical abundance of microbial life in relation to
ecology and biogeochemistry of Earth systems.

\subsubsection{General Questions}\label{general-questions}

\paragraph{What were the main questions being
asked?}\label{what-were-the-main-questions-being-asked}

What is the number of prokaryotic cells in different environments on
Earth? Which areas are the prokaryotes the most abundant? What is the
amount of prokaryotic cellular carbon in different habitats? How much of
the total Earth carbon content is contributed by the prokaryotic
cellular carbon? What is the turn over rate of the prokaryotic
population and how does it contribute to the mutation rate and therefore
diversity in prokaryotes?

\paragraph{What were the primary methodological approaches
used?}\label{what-were-the-primary-methodological-approaches-used}

The primary methodological approaches are estimations which is to take
small samples to represent the habitat. Additionally, they use the cell
density, volume, and cellular carbon of prokaryotes to calculate total
number of cells in a habitat. Specific calculation of each habitat
methodological approaches: For aquatic environment (polar region)- they
estimated number of prokaryotes based on the mean cell numbers reported
in previous literature Delille and Rosiers, and the mean areal extent of
seasonal ice. For soil- they estimated the number of prokaryotes based
on detained direct counts from a coniferous forest ultisol. They used
this estimate and applied it to all the forest soils. For subsurface-
they used few direct enumerationd of subsurface prokaryotes. For deeper
sediments, they extrapolated the number of prokaryoted based on a
formula. Another approach that they used is based on the porosity of the
terrestrial subsurface, and total pore space occupied by prokaryotes.
Also they used the number of unattached prokaryoted calculated from
groundwater data from previous studies. To calculate Carbon content-
They used the prokaryotic cell numbers, and then they assumed that the
amount of the cellular carbon is half of the dry weight. They also
assumed the amount of carbon production should be 4 times the the carbon
content based on the efficiency of carbon assimilation. They used this
information to make an assumption about the turn over rate of
prokaryotes.

\paragraph{Summarize the main results or
findings.}\label{summarize-the-main-results-or-findings.}

Most of the prokaryotes reside in open ocean (1.2 x 10\^{}29), soil (1.2
x 10\^{}29), oceanic (3.5 x 10\^{}30) and terrestrial subsurfaces
(0.25-2.5 x 10\^{}30). The number of prokaryotes on Earth is estimated
to be 4-6 x 10\^{}30 cells. The amount of prokaryotic cellular carbon on
Earth is estimated to be 350-550 Pg. Including prokaryotic carbon in
global models will nearly double the amount of carbon stored in living
organisms.Prokaryotes contain 85-130 Pg of N and 9-14 Pg of P, which is
about 10-fold more nutrients than found in plants.The average turn over
times of prokaryotes is 6-25 days in the upper 200 m of the open ocean,
0.8 yr in the ocean below 200 m, 2.5 yr in soil, and 1-2 x 10\^{}3 yr in
subsurface. The cellular production rate for prokaryotes is estimated to
be 1.7 x 10\^{}30 cells/year.Prokaryotes exist in plants, air, and leafs
but the number is much lower compared to the large resivors. Due to the
abundance of prokaryotes and high turn over rates, there is an enormous
genetic diversity among prokaryotes.

\paragraph{Do new questions arise from the
results?}\label{do-new-questions-arise-from-the-results}

How do we define prokaryotes? what is the composition of microbes in
each compartment? What are the measures that have to be taken into
account in the phylogenic analyses to distinguish prokaryotes from
eukaroytes? Is there any better measuring techniques in place that can
lower the assumptions and error? How reliable are the numbers that are
mentioned in this article? How does the turn over rate of the
prokaryotes affect the nutrient cycles and impact other organisms?

\paragraph{Were there any specific challenges or advantages in
understanding the paper (e.g.~did the authors provide sufficient
background information to understand experimental logic, were methods
explained adequately, were any specific assumptions made, were
conclusions justified based on the evidence, were the figures or tables
useful and easy to
understand)?}\label{were-there-any-specific-challenges-or-advantages-in-understanding-the-paper-e.g.did-the-authors-provide-sufficient-background-information-to-understand-experimental-logic-were-methods-explained-adequately-were-any-specific-assumptions-made-were-conclusions-justified-based-on-the-evidence-were-the-figures-or-tables-useful-and-easy-to-understand}

The authors didn't explain the calculations well and mostly provided us
with the final numbers.It would be beneficial to have a supplementary
attached which was going in to details of errors and how reliable are
the numbers reported in the article.The authors didn't state their
assumptions clearly. Additionally, the authors used a lot of data and
methods from previous literatures but they didn't explain much about
them

\subsection{\texorpdfstring{Evidence Worksheet\_02 ``Life and the
Evolution of Earth's
Atmosphere''}{Evidence Worksheet\_02 Life and the Evolution of Earth's Atmosphere}}\label{evidence-worksheet_02-life-and-the-evolution-of-earths-atmosphere}

\href{http://science.sciencemag.org/content/296/5570/1066/tab-article-info}{Kasting
\& Siefert 2003}

\subsubsection{Learning objectives:}\label{learning-objectives-1}

Comment on the emergence of microbial life and the evolution of Earth
systems

\subsubsection{General Questions}\label{general-questions-1}

\begin{itemize}
\item
  Indicate the key events in the evolution of Earth systems at each
  approximate moment in the time series. If times need to be adjusted or
  added to the timeline to fully account for the development of Earth
  systems, please do so.

  \begin{itemize}
  \item
    4.6 billion years ago : Inner planets received carbon and water
    vapour. star system developed due to super nova explotions. Moon
    formed and spin and tilt of Earth evolved leading to present day and
    night cycles and seasons.
  \item
    4.5-4.1 billion years ago: Due to early sun being weak, high levels
    of CO2 increased the temperature.
  \item
    4.2 billion years ago : Evidence of life (Isotopes of C preserved in
    grafite). Oldest Rock (Coherent assamblages of mineral).
  \item
    3.8 billion years ago : Oldest sedimentary rocks and methanogenesis-
    oceans and weathering- maybe the oldest known sediments.
  \item
    3.5 billion years ago : Photosynthesis by Cyanobacteria and
    microfossils and stromatolites present. Stromatolites are
    organosedimentary structures produced by microbial trapping.
  \item
    2.7 billion years ago : Great oxidation event: responsible for
    glaciation. Here the planet would have been brown because of the
    methanogens to keep the planet warm because the sun was less
    illuminant.
  \item
    2.2 billion years ago : Increase in biological production.
  \item
    1.7 billion years ago : Appearance of eukaryotes in the form of
    algea.
  \item
    1.1 billion years ago : Snowball Earth
  \item
    550,000 years ago : Cambrian explosion where larger animals appear.
    Land plants also observed which once again increase the oxygen
    concentration in atmosphere.
  \item
    200,000 years ago : Gigantic organims appear. Permian extinction
    when 95\% of species extinct.
  \end{itemize}
\item
  Describe the dominant physical and chemical characteristics of Earth
  systems at the following waypoints:

  \begin{itemize}
  \item
    Hadean : During this time, there was a high concentration of CO2 to
    keep the Earth warm as the sun was 30\% less illuminant. Earth was
    also very hot. Spin and tilt of the Earth evolved due to the
    formation of moon. Star system formed due to super nova explosion.
  \item
    Archean : High concentration of methane produced by methanogens to
    keep th Earth warm. Some O2 was present in the atmosphere due to the
    photosynthesis of Cyanobacteria.
  \item
    Proterozoic : CO2 produced as a result of the raction between oxygen
    and methane. This was responsible for making the Earth cold due to
    the decrease in the green house effect which lead to glaciation.
  \item
    Phanerozoic : Increase in atmospheric oxygen concentration due to
    increase in land plants. Formation of coal deposits die to the death
    of the organims.
  \end{itemize}
\end{itemize}

\subsection{\texorpdfstring{Evidence Worksheet\_ 03 ``The
Anthropocene''}{Evidence Worksheet\_ 03 The Anthropocene}}\label{evidence-worksheet_-03-the-anthropocene}

\href{https://www.nature.com/articles/461472a}{Rockstrom et al 2009}

\subsubsection{Learning Objectives}\label{learning-objectives-2}

Evaluate human impacts on the ecology and biogeochemistry of Earth
systems.

\subsubsection{Specific questions}\label{specific-questions}

\paragraph{What were the main questions being
asked?}\label{what-were-the-main-questions-being-asked-1}

How humans can operate safely and prevent their activities to cause
environmental changes? What is the threshold within which humans should
operate? How can boundaries for various processes be combined and
connected? How have humans impacted Earth?

\paragraph{What were the primary methodological approaches
used?}\label{what-were-the-primary-methodological-approaches-used-1}

The author develops his arguments through citing multiple authors and
books in the paper. Using Planetary boundaries. For defining these
boundaries, the author cite various literatures on the fossil records
about the extinction rates and atmospheric carbon dioxide levels.

\paragraph{Summarize the main results or
findings.}\label{summarize-the-main-results-or-findings.-1}

Unacceptable environmental changes can be generated if the threshold is
crossed for the processes. 9 processes were discussed that need
planetary boundaries: change in land use, atmospheric aerosol loading,
ocean acidification, global fresh water use, stratospheric ozone
depletion, rate of biodiversity loss, climate change, and interference
with nitrogen and phosphorus cycles. Some of the boundaries for these
processes have been passed such as for climate change. Additionally, the
boundaries for some other processes are soon to be approached such as
global freshwater use. If these thresholds are not crossed by humans,
they can purse long-term social and economic developments. Changes in
the atmospheric carbon dioxide should not exceed 350 parts per million
by volume.

\paragraph{Do new questions arise from the
results?}\label{do-new-questions-arise-from-the-results-1}

Are the proposed models for setting boundaries reliable? Are the
threshold discussed take into account the increasing rate of human
activities on Earth? Does these threshold values vary from location to
location on Earth?

\paragraph{Were there any specific challenges or advantages in
understanding the paper (e.g.~did the authors provide sufficient
background information to understand experimental logic, were methods
explained adequately, were any specific assumptions made, were
conclusions justified based on the evidence, were the figures or tables
useful and easy to
understand)?}\label{were-there-any-specific-challenges-or-advantages-in-understanding-the-paper-e.g.did-the-authors-provide-sufficient-background-information-to-understand-experimental-logic-were-methods-explained-adequately-were-any-specific-assumptions-made-were-conclusions-justified-based-on-the-evidence-were-the-figures-or-tables-useful-and-easy-to-understand-1}

The authors provided enough background information to understand the
paper. However, the authors didn't include any figures to describe their
proposed models which would have helped the readers further to
understand the author's rationals. Also, sometimes the authors claim
facts without providing evidence.

\subsection{\texorpdfstring{Problem set\_01 for ``Prokaryotes: The
Unseen
Majority''}{Problem set\_01 for Prokaryotes: The Unseen Majority}}\label{problem-set_01-for-prokaryotes-the-unseen-majority}

\subsubsection{Learning objective}\label{learning-objective}

Describe the numerical abundance of microbial life in relation to the
ecology and biogeochemistry of Earth systems.

\subsubsection{Specific questions}\label{specific-questions-1}

\paragraph{What are the primary prokaryotic habitats on Earth and how do
they vary with respect to their capacity to support life? Provide a
breakdown of total cell abundance for each primary habitat from the
tables provided in the
text.}\label{what-are-the-primary-prokaryotic-habitats-on-earth-and-how-do-they-vary-with-respect-to-their-capacity-to-support-life-provide-a-breakdown-of-total-cell-abundance-for-each-primary-habitat-from-the-tables-provided-in-the-text.}

It has been indicated that most of the prokaryotes reside in three large
habitats: seawater, soil and the sediment/soil subsurface. Aquatic-
1.181 x 10\^{}29 cells. Soil- 2.556 x 10\^{}29. Subsurface sediments-
3.8 x 10\^{}30 (oceanic subsurface: 3.5 x 10\^{}30 and terrestrial
subsurface: 0.25-2.5 x 10\^{}30).

\paragraph{What is the estimated prokaryotic cell abundance in the upper
200 m of the ocean and what fraction of this biomass is represented by
marine cyanobacterium including Prochlorococcus? What is the
significance of this ratio with respect to carbon cycling in the ocean
and the atmospheric composition of the
Earth?}\label{what-is-the-estimated-prokaryotic-cell-abundance-in-the-upper-200-m-of-the-ocean-and-what-fraction-of-this-biomass-is-represented-by-marine-cyanobacterium-including-prochlorococcus-what-is-the-significance-of-this-ratio-with-respect-to-carbon-cycling-in-the-ocean-and-the-atmospheric-composition-of-the-earth}

The estimated prokaryotic cell abundance is 3.6 x 10\^{}28 in the upper
200 m at cellular density of 5 x 10\^{}5 cells/ml. The average cellular
density of the autotrophic marine. cyanobacteria and
\emph{prochlorococcus spp.} in the upper 200 m is 4 x 10\^{}4 cells/ml.
The fraction is : (4x 10\^{}4)/ (5x 10\^{}5) = 8\%. This means that 8\%
of the autotrophic prokaryotes in charge of carbon cycle in the ocean to
provide organic carbon for the majority of the hetrotrophic prokaryotes.

\paragraph{What is the difference between an autotroph, heterotroph, and
a lithotroph based on information provided in the
text?}\label{what-is-the-difference-between-an-autotroph-heterotroph-and-a-lithotroph-based-on-information-provided-in-the-text}

Autotrophs is ``self- nourishing'', fix inorganic carbon (CO2) into
biomass using light source as energy. Heterotroph: assimilate organic
carbon- Uses organic substrates for both Carbon and energy source.
Litotroph: use inorganic substrates for C source and energy source.

\paragraph{Based on information provided in the text and your knowledge
of geography what is the deepest habitat capable of supporting
prokaryotic life? What is the primary limiting factor at this
depth?}\label{based-on-information-provided-in-the-text-and-your-knowledge-of-geography-what-is-the-deepest-habitat-capable-of-supporting-prokaryotic-life-what-is-the-primary-limiting-factor-at-this-depth}

Subsurface: Teressterial and marine: \textasciitilde{}4 Km deep in the
subsurface there is life. Deepest point in the ocean is Mariana's
Trench: 10.9 km down. Therefore 4 km lower than Mariana's Trech is 14.9
km deep down that life could exist.Temperature is the limiting factor-
it is about 125 C. Change in temperature is 22 C/km.

\paragraph{Based on information provided in the text your knowledge of
geography what is the highest habitat capable of supporting prokaryotic
life? What is the primary limiting factor at this
height?}\label{based-on-information-provided-in-the-text-your-knowledge-of-geography-what-is-the-highest-habitat-capable-of-supporting-prokaryotic-life-what-is-the-primary-limiting-factor-at-this-height}

Mount Everst is the highest point which is about 8.8 km above the sea
level. Also microbes have been found to be transported up to 77 km.
However the limiting factor is the ionizing radiation and lack of
nutrients and moisture. Therefore the microbes might not be metabolizing
and maybe spores.

\paragraph{Based on estimates of prokaryotic habitat limitation, what is
the vertical distance of the Earth's biosphere measured in
km?}\label{based-on-estimates-of-prokaryotic-habitat-limitation-what-is-the-vertical-distance-of-the-earths-biosphere-measured-in-km}

From the top of mount Everest (8.8 km up) to the bottom of Mariana's
Trench (10.9 km down) and 4 km deeper in to the sediment is 24 km.

\paragraph{How was annual cellular production of prokaryotes described
in Table 7 column four determined? (Provide an example of the
calculation)}\label{how-was-annual-cellular-production-of-prokaryotes-described-in-table-7-column-four-determined-provide-an-example-of-the-calculation}

Population x turn overtime /year Ex. for marine habitats: (3.6 x
10\^{}28 cells x 365 days/ 16 turnover = 8.2 x 10\^{}29 cells/ yr)

\paragraph{What is the relationship between carbon content, carbon
assimilation efficiency and turnover rates in the upper 200m of the
ocean? Why does this vary with depth in the ocean and between
terrestrial and marine
habitats?}\label{what-is-the-relationship-between-carbon-content-carbon-assimilation-efficiency-and-turnover-rates-in-the-upper-200m-of-the-ocean-why-does-this-vary-with-depth-in-the-ocean-and-between-terrestrial-and-marine-habitats}

Carbon efficiency- assumed to be 20\%- only this percent is being
assimilated. About 5-20 fg on average C in prokaryotic cells (used 20 fg
for calculations): About 20 x 10\^{}-30 pg/ cell.

3.6 x 10\^{} 28 cells x 20 x10\^{}-30 pg/ cell = 0.72 pg carbon in
marine heterotrophs.

They used multiplier of 4 instead of 5 for 20\%-\textgreater{} 4 x 0,72=
2.88 pg/ yr.

51 pg/ yr (carbon in photic zone) x 85\% consumed = 43 pg C

43 pg C/ yr / 2.88 pg /yr = 14.9 or 1 turnover every 24.5 days. Half of
the carbon of the planet is among the prokaryotes. This number vary with
depth in ocean and between terrestrial and marine habitats due to the
change in microbial population.

\paragraph{How were the frequency numbers for four simultaneous
mutations in shared genes determined for marine heterotrophs and marine
autotrophs given an average mutation rate of 4 x 10-7 per DNA
replication? (Provide an example of the calculation with units. Hint:
cell and generation cancel
out)}\label{how-were-the-frequency-numbers-for-four-simultaneous-mutations-in-shared-genes-determined-for-marine-heterotrophs-and-marine-autotrophs-given-an-average-mutation-rate-of-4-x-10-7-per-dna-replication-provide-an-example-of-the-calculation-with-units.-hint-cell-and-generation-cancel-out}

For heterotrophes: (4 x 10\^{}-7 mutations/ generations) \^{}4 = 2.56 x
10\^{}-26 mutation/ generation

365 days/ 16=22.8 turnover/ year

3.6 x 10\^{}28 cells in marine x 22.8 turn over/year= 8.2 x 10\^{}29
cells/ year.

2.56 x 10\^{}-26 mutation/generation x 8.2 x 10\^{}29 cells/ year x 1
year/ 12 months x 1 month/ 28 days x 1 day/ 24 hours = 2.603 mutation/
hr

Frequency: 1/ 2.603 = 0.4

For autothrophes: (4 x 10\textsuperscript{-7)}4 x 7.1 x 10\^{}29 turn
over/ year x 1 year/ 8064 hr= 2.553 mutation/ hr.

Frequency: 1/ 2.553\textasciitilde{} 0.5

\paragraph{Given the large population size and high mutation rate of
prokaryotic cells, what are the implications with respect to genetic
diversity and adaptive potential? Are point mutations the only way in
which microbial genomes diversify and
adapt?}\label{given-the-large-population-size-and-high-mutation-rate-of-prokaryotic-cells-what-are-the-implications-with-respect-to-genetic-diversity-and-adaptive-potential-are-point-mutations-the-only-way-in-which-microbial-genomes-diversify-and-adapt}

Mutations in prokaryotic cells are the major source of diversity and one
of the essential factors in the formation of novel species.
Additionally, due to mutations, prokaryotes can be selected for and
adapt to variety of environments. Point mutations are not the only way
in which microbial genomes diversify and adapt. Horizontal gene transfer
(HGT), insertions, deletions, and vertical gene transfer are other ways
that microbial genome can diversity and adapt.

\paragraph{What relationships can be inferred between prokaryotic
abundance, diversity, and metabolic potential based on the information
provided in the
text?}\label{what-relationships-can-be-inferred-between-prokaryotic-abundance-diversity-and-metabolic-potential-based-on-the-information-provided-in-the-text}

Large population size and rapid replication of prokaryotes implies that
extremely rare events can occur frequently. High mutation leads to
increase in diversity and therefore increase in metabolic potential in
response to stress and selective pressure.

\subsection{\texorpdfstring{Problem set\_02 ``Microbial
Engines''}{Problem set\_02 Microbial Engines}}\label{problem-set_02-microbial-engines}

\subsubsection{Learning objectives:}\label{learning-objectives-3}

Discuss the role of microbial diversity and formation of coupled
metabolism in driving global biogeochemical cycles.

\subsubsection{Specific Questions:}\label{specific-questions-2}

\paragraph{What are the primary geophysical and biogeochemical processes
that create and sustain conditions for life on Earth? How do abiotic
versus biotic processes vary with respect to matter and energy
transformation and how are they
interconnected?}\label{what-are-the-primary-geophysical-and-biogeochemical-processes-that-create-and-sustain-conditions-for-life-on-earth-how-do-abiotic-versus-biotic-processes-vary-with-respect-to-matter-and-energy-transformation-and-how-are-they-interconnected}

Geophysical: Tectonics and atmospheric photochemical
processes-\textgreater{} supply substrates and remove products. They
allow elements and molecules to interact with each other, and chemical
bonds to form and break in cyclical manner.

Biogeochemical: The biological fluxes of 6 major elements- H,C,N,O,S,
and P that constitute major building blocks for all biological
macromolecules. Geochemical reactions are based on acid/ base chemistry.
The biogeochemical fluxes are driven largely by microbes that catalyze
thermodynamically constrained reactions. Additionally, volcanism and
rock weathering play a role in resupply of C, S, and P.

Abiotic reactions: are based on acid/base chemistry (transfer of protons
without electrons).

Biotic reactions: redox reaction that is dependant on the successive
transfers of electrons and protons from a relatively limited set of
chemical elements.

The way that the biotic and abotic processes are interconnected is that
biogeochemical cycles have evolved on a planetary scale to form a set of
nested abiotiotically driven acid/base reactions and biologically driven
redox reactions that set lower limits on external energy that are
required to sustain these cycles.

\paragraph{Why is Earth's redox state considered an emergent
property?}\label{why-is-earths-redox-state-considered-an-emergent-property}

This is because feedbacks between the evolution of microbial metabolic
and geochemical processes create the average redox condition of the
oceans and atmosphere. Therefore, Earth's redox state is an emergent
property of microbial life on a planetary scale.

\paragraph{How do reversible electron transfer reactions give rise to
element and nutrient cycles at different ecological scales? What
strategies do microbes use to overcome thermodynamic barriers to
reversible electron
flow?}\label{how-do-reversible-electron-transfer-reactions-give-rise-to-element-and-nutrient-cycles-at-different-ecological-scales-what-strategies-do-microbes-use-to-overcome-thermodynamic-barriers-to-reversible-electron-flow}

Microbes carry genes that encode the multimerci protein complexes
(machinery) responsible for redox chemistry (transfers of electron and
protons) half-cells that form the basis of the major energy transducting
pathways. In order to overcome thermodynamic barriers to reversible
electron flow, microbes use identical or near-identical pathways for
forward and reverse reactions to maintain cycles. That in one direction
the process is oxidative, dissimilatory and energy producing, and in the
other direction it is reductive, assimilatory and energy consuming. This
process requires the synergistic cooperation of multispecies
assemblages, a phenomenon that is typical for most biogeochemical
transformations.

\paragraph{\texorpdfstring{Using information provided in the text,
describe how the nitrogen cycle partitions between different redox
``niches'' and microbial groups. Is there a relationship between the
nitrogen cycle and climate
change?}{Using information provided in the text, describe how the nitrogen cycle partitions between different redox niches and microbial groups. Is there a relationship between the nitrogen cycle and climate change?}}\label{using-information-provided-in-the-text-describe-how-the-nitrogen-cycle-partitions-between-different-redox-niches-and-microbial-groups.-is-there-a-relationship-between-the-nitrogen-cycle-and-climate-change}

For N2 to become accessible and be used in the synthesis of nucleic
acids and proteins, N2 should be changed to NH4+ via nitrogen fixation.
However, the enzyme responsible for fixation is inhibited by Oxygen. In
the presence of oxygen, NH4+ can be oxidized to nitrate in a two-stage
pathway, initially requiring a specific group of Bacteria or archea that
oxidize ammonia to NO2- (via hydroxyamine), which is subsequently
oxidized to NO3- by a different suite of nitrifying bacteria. The small
differences in redox potential in the redox reactions is used by
nitrifiers to reduce CO2 to organic matter. Finally, in the absence of
oxygen, a third set of opportunistic microbes uses NO2- and NO3- as
electron acceptors in anaerobic oxidation of organic matter. This
pathway will ultimately form N2.

The major cuase of climate change is the green house effect due to
increase in the concentration of green house gasses in the Earth's
atmosphere such as CO2. The nitrogen cycle affects the climate change,
as CO2 is reduced to organic matter by the oxidation of NH4+ to NO2 and
NO2 to NO3. Therefore this reduces atmospherinc CO2.

\paragraph{What is the relationship between microbial diversity and
metabolic diversity and how does this relate to the discovery of new
protein families from microbial community
genomes?}\label{what-is-the-relationship-between-microbial-diversity-and-metabolic-diversity-and-how-does-this-relate-to-the-discovery-of-new-protein-families-from-microbial-community-genomes}

The more diverse that the microbial community is, the more diverse are
the metabolic activities that are observed in the community. The number
of new protein families found are increasing linearly with the number of
new genome sequenced. Also the discovery rate for new proteins is
linearly even when the known number of protein sequences is increased
threefold. This indicates that the journey of cataloguing extant protein
sequence space is relatively new due to the limitless evolutionary
diversity in nature.

\paragraph{On what basis do the authors consider microbes the guardians
of
metabolism?}\label{on-what-basis-do-the-authors-consider-microbes-the-guardians-of-metabolism}

Environmental selection on the microbial phenotype leads to the
evolution of the boutique genes that ultimately protect the metabolic
pathway. Dispersal of the core planetary gene set whether by vertical or
horizontal gene transfer, has allowed wide variety of organisms to
become guardians of metabolism simultaneously. That is even the pathway
in a specific operational taxonomic unit does not survive an
environmental perturbation the unit will go extinct but the pathway has
a strong chance of survival in other units.

\subsection{Writing Assignment 1}\label{writing-assignment-1}

``Microbial life can easily live without us; we, however, cannot survive
without the global catalysis and environmental transformations it
provides.'' Do you agree or disagree with this statement? Answer the
question using specific reference to your reading, discussions and
content from evidence worksheets and problem sets.

Microorganism, although can't be seen with naked eye, are ubiquities.
Microbes have been the earliest habitants of the Earth and have
enormously evolved from then to adapt to the ever-changing conditions on
Earth. While humans are exploiting the Earth resources and contribution
to it's degradation, microbes are playing an extremely important role in
the environmental transformation to make Earth a habitable place for
humans. It can be reasonably argued that humans cannot survive without
the global catalysis and environmental transformation of microbes.

Microbes play an essential role in catalyzing major biogeochemical
fluxes of H, C, N, O, and S that humans cannot mimic it but depend on it
for survival. These fluxes are constituted of thermodynamically
constrained redox reactions (1). In order to make these redox processes
thermodynamically favorable, there is a requirement for the synergistic
cooperation of multispecies assemblages (1). Therefore, microbes in a
community cooperate by catalyzing coupled forward and reverse reactions,
where one is oxidative and dissimilatory and the other is reductive and
assimilatory (1). Since these metabolic pathways are coupled, then
biosphere can approach ``self-sustaining element cycling on the time
scales of centuries to millennia'' (1). Additionally, microbes have been
characterized as ``guardians of metabolism'', due to the fact that core
genes of biogeochemical cycles are transferred among members of the
community through horizontal gene transfer (1). Therefore, this allows
variety of different organisms in diverse environmental contexts to
retain the genes of fundamental redox processes and protect the
metabolic pathway even if the individual taxonomic unit goes extinct
(1).

It should be noted that humans do not have the knowledge of all the
interconnections between the biogeochemical processes (1). Therefore, it
is plausible to suggest that even if humans were able to intervene and
take over the biogeochemical processes, they would not be able to
complete the cycle. An example is the development of synthetic
fertilizers (Haber-Bosch process) that would boost crop yields (2).
Nitrifying bacteria converts the NH4+ in the fertilizer to the highly
mobile NO3-, which would run into lakes and rivers and cause
eutrophication of coastal zones (2). Additionally, denitrification under
anoxic conditions of soil will lead to the production of N2O, which is
lost to the atmosphere and further contributes to the greenhouse gas
effect caused by humans (2). Microbes have developed the nitrogen cycle
about 2.7 billion years ago with natural feedback and controls, but
humans the imbalance of nitrogen cycle by developing synthetic
fertilizers.

Because of the human activities such as combustion of fossil fuels and
deforestation, dramatic increase in atmospheric CO2 concentration has
been observed (3). Not only this increase in CO2 will lead to direct
ecological disruption, but also it has the largest impact on the climate
system (3). Eventually the elevation of temperature due to increase in
atmospheric CO2 combined with additional impacts of human activities
such as pollution will make conditions on Earth intolerable and exceed
the tolerance of ecosystem to adapt to increase in temperature (3). This
can lead to mass extinction. Although one can argue that humans might
take an active role to engineer the climate and reverse the perturbation
that they have caused in the first place. It should be noted that the
process of ``geoengineering'', if successful, is a slow process that
might not meet the demands in time (3).

On the other hand, microbes are contributing to large fluxes of CO2
consumption. The major contributor to this negative flux is oxygenic
photosynthetic cyanobacteria, which are believed to be responsible for
the initial increase in atmospheric O2 about 2.3 billion years ago (4,
5). In oxygenic photosynthesis, Rubisco captures CO2 from the
atmosphere-ocean, H2O act as an electron donor, and light is used and an
energy source. Additional examples of pathways where microbes are
involved in assimilation of CO2 into organic matter are reductive citric
acid cycle and reductive acetyl-CoA pathway (1). Another way that
microbes contribute to the flux of CO2 consumption is through promoting
plant growth by fixing nitrogen (6). Plants not only contribute to
decrease in atmospheric CO2 through photosynthesis, and increase in
atmospheric O2, they also serve as a food source for humans and animals.
Therefore, in the absence of microorganisms, primary production would be
negatively impacted which further hugely affect the terrestrial and
oceanic food webs. It should also be noted that majority of O2 that we
breath is produced by photosynthetic bacteria in oceans such as
prochlorococcus and Synechoccocus (6).

Prokaryotes have extremely large population size and rapid growth rate,
which leads to higher mutation events and therefore immense capacity for
genetic diversity (7). The genetic diversity has enabled prokaryotes to
adapt to wide range of habitants, and even survive some of the harshest
conditions such as hydrothermal vents (4). If conditions on Earth become
inhabitable, microbes can be selected for, adapt and survive and
potentially change the composition of Earth's atmosphere as they have
been doing since the origin of life. Looking back over the history of
Earth, about 3 billion years ago, the sun was much dimmer compared to
now and therefore methanogen bacteria were contributing to green house
effect by producing CH4 which prevented Archean Earth from freezing and
blocked UV light (5). About 2.4 billion years ago, microbes,
specifically cyanobacteria, caused Earth great oxygenation event, which
eventually lead to the presence of plants, animals and even humans on
Earth (5). It should not be taken lightly, that microbes survived some
of the harshest environmental conditions such as Snowball Earth.
Therefore, if another such events are going to make Earth inhabitable,
based on the evidence we have, microbes are going to be the only
survivors.

Not only microbes are essential to reverse the green house effect caused
by humans, they can also take parts in pollution degradation that has
been accumulating on Earth due to destructive human activities. As an
example, 1.3 million liters of crude oil (petroleum), which is composed
of highly complex mixtures of organic compounds, enter the environment
each year (9). The major contribution to the release of crude oil is
illegal activities such the discharge of ballast water and oil residues
in addition to oil spillage accidents (9). Without the activity of the
hydrocarbon-degrading microorganism, the ocean would have been covered
in an oily layer (9). Examples of obligate hydrocarbon degraders include
the genera Alcanivorax, Oleispira, and Cycloclasticus (9). Therefore,
without microbes, waste management would be another problem that humans
must deal with on top of green house effect.

In conclusion, not only microbes are contributing to the major
biogeochemical cycles and metabolic pathways, they are also aiding
humans in reversing the negative effects that they have caused by
overexploitation of resources. I strongly believe that without microbes,
the process of Earth becoming an uninhabitable place would accelerate in
a much faster speed. Therefore, microbes are essential, and in order to
save the Earth and its habitats dual intervention by microbes and humans
is needed.

References

\begin{enumerate}
\def\labelenumi{\arabic{enumi}.}
\tightlist
\item
  Falkowski PG, Fenchel T, Delong EF. 2008. The Microbial Engines That
  Drive Earths Biogeochemical Cycles. Science 320:1034-1039.
\item
  Canfield DE, Glazer AN, Falkowski PG. 2010. The Evolution and Future
  of Earth's Nitrogen Cycle. Science 330:192-196.
\item
  Schrag DP. 2012. Geobiology of the Anthropocene. Fundamentals of
  Geobiology 425-436.
\item
  Nisbet EG, Sleep NH. 2001. The habitat and nature of early life.
  Nature 409:1083-1091.
\item
  Kasting JF, Siefert JL. 2002. Life and the Evolution of Earth's
  Atmosphere. Life and the Evolution of Earth's Atmosphere
  296:1066-1068.
\item
  Gilbert JA, Neufeld JD. 2014. Life in a World without Microbes. PLoS
  Biology 12.
\item
  Whitman WB, Coleman DC, Wiebe WJ. 1998. Prokaryotes: The unseen
  majority. Proceedings of the National Academy of Sciences
  95:6578-6583.
\item
  Brooijmans RJW, Pastink MI, Siezen RJ. 2009. Hydrocarbon-degrading
  bacteria: the oil-spill clean-up crew. Microbial Biotechnology
  2:587-594.
\end{enumerate}

\subsection{Module 1 References:}\label{module-1-references}

\begin{enumerate}
\def\labelenumi{\arabic{enumi}.}
\item
  Falkowski PG, Fenchel T, Delong EF. 2008. The Microbial Engines That
  Drive Earths Biogeochemical Cycles. Science 320:1034-1039.
  \href{http://science.sciencemag.org/content/320/5879/1034.long}{PMID18497287}
\item
  Kasting JF, and Siefert JL. 2002. Life and the evolution of Earth's
  atmosphere. Science. 296(5570):1066-1068.
  \href{https://www.ncbi.nlm.nih.gov/pubmed/12004117}{PMID12004117}
\item
  Leopold A. 1949. The Land Ethic. In A Sand County Almanac. Oxford
  University Press. London.
\item
  Zehnder A.J.B. 1988. Biology of Anaerobic Microorganisms.
\item
  Nisbet EG, Sleep NH. 2001. The habitat and nature of early life.
  Nature 409:1083-1091. (\url{https://www.nature.com/articles/35059210})
\item
  Whitman WB, Coleman DC, Wiebe WJ. 1998. Prokaryotes: The unseen
  majority. Proceedings of the National Academy of Sciences
  95:6578-6583.
  \href{https://www.ncbi.nlm.nih.gov/pmc/articles/PMC33863/}{PMC33863}
\item
  Waters CN. 2016. The Anthropocene is functionally and
  stratigraphically distinct from the Holocene. Science 351:137--147.
  \href{https://www.ncbi.nlm.nih.gov/pubmed/26744408}{PMID26744408}
\item
  Schrag DP. 2012. Geobiology of the Anthropocene. Fundamentals of
  Geobiology 425--436.
  (\url{https://onlinelibrary.wiley.com/doi/10.1002/9781118280874.ch22})
\item
  Kallmeyer J, Pockalny R, Adhikari RR, Smith DC, and D'Hondt S. 2012.
  Global distribution of microbial abundance and biomass in subseafloor
  sediment. Proc Natl Acad Sci USA. 109(40):16213-16216.
  \href{https://www.ncbi.nlm.nih.gov/pubmed/22927371}{PMID22927371}
\item
  Mooney C. 2016. Scientists say humans have now brought on an entirely
  new geologic epoch. The Washington Post 1--5.
  (\url{https://www.washingtonpost.com/news/energy-environment/wp/2016/01/07/scientists-say-humans-have-now-brought-on-an-entirely-new-geologic-epoch/?utm_term}=.a25428157ae9)
\item
  Rockstrom J et al. 2009. A safe operating space for humanity. Nature
  461, 472--475. (\url{https://www.nature.com/articles/461472a})
\item
  Achenbach J. 2012. Spaceship Earth: A new view of environmentalism.
  The Washington Post. WP Company.
  (www.washingtonpost.com/national/health-science/spaceship-earth-a-new-view-of-environmentalism/2011/12/29/gIQAZhH6WP\_story.html)
\item
  Canfield DE, Glazer AN, Falkowski PG. 2010. The Evolution and Future
  of Earth's Nitrogen Cycle. Science 330:192--196.
  \href{https://www.ncbi.nlm.nih.gov/pubmed/20929768}{PMID20929768}
\item
  Falkowski P, Scholes RJ, Boyle E, Canadell J, Canfield D, Elser J,
  Gruber N, Hibbard K, Högberg P, Linder S, Mackenzie FT, Moore B 3rd,
  Pedersen T, Rosenthal Y, Seitzinger S, Smetacek V, and Steffen W.
  2000. The Global Carbon Cycle: A Test of Our Knowledge of Earth as a
  System. Science 290:291--296.
  \href{https://www.ncbi.nlm.nih.gov/pubmed/11030643}{PMID11030643}
\end{enumerate}

\section{Module 2:}\label{module-2}

\subsection{\texorpdfstring{Evidence Worksheet\_ 04 ``Bacterial
Rhodopsin Gene
Expression''}{Evidence Worksheet\_ 04 Bacterial Rhodopsin Gene Expression}}\label{evidence-worksheet_-04-bacterial-rhodopsin-gene-expression}

{[}Martinez et al
2007{]}(\url{https://www.ncbi.nlm.nih.gov/pmc/articles/PMC1838496/}

\subsubsection{Learning Objectives}\label{learning-objectives-4}

\begin{itemize}
\item
  Discuss the relationship between microbial community structure and
  metabolic diversity
\item
  Evaluate common methods for studying the diversity of microbial
  communities
\item
  Recognize basic design elements in metagenomic workflows
\end{itemize}

\subsubsection{Specific questions}\label{specific-questions-3}

\paragraph{What were the main questions being
asked?}\label{what-were-the-main-questions-being-asked-2}

What are the fully PR photosystem genetics and biochemistry?

What are the various functions and physiological roles of diverse marine
microbial PRs?

\paragraph{What were the primary methodological approaches
used?}\label{what-were-the-primary-methodological-approaches-used-2}

Surveyed a marine picoplankton large-insert genomic library for
recombinant clones expressing PR photosystems in vivo.

Screened large-insert DNA libraries derived from marine picoplankton for
visibly detectable PR-expressing phenotypes.

Exploited transient increases in vector copy number that significantly
enhanced the sensitivity of phenotypic detection. (Used the copy-control
system present in the fosmid vector to allow a controlled transition
from one copy per cell to multiple).

Genomic analyses of candidate PR photosystem-expressing clones

Genetic and phenotypic analysis of PR photosystem

Light-activated proton translocation

Measuring light-induced changes in ATP levels in the
PR-photosystem-containing clones and PR- mutant derivatives by using a
luciferase-based assay.

\paragraph{Summarize the main results or
findings.}\label{summarize-the-main-results-or-findings.-2}

The PR photosystems among diverse microbial taxa are ubiquitous.

A single genetic event can results in the acquisition of phototrophic
capabilities in an otherwise chemotrophic microorganisms.

The genes of the proton pump has been evolved to respond to the amount
of the light reaching the ocean level. This is called spectro-toning.

Plasmid have another Origin of replication- rep proteins bind to them
that due to local melting giving access to polymerase. Mutant of Rep
copy under inducible arabinose, mutant Rep protein keeps going and
wouldn't dimerize to form higher forms that prevent them from
polymerizing.

\paragraph{Do new questions arise from the
results?}\label{do-new-questions-arise-from-the-results-2}

What other systems might get activated in a microorganism as a result of
single mutation?

\paragraph{Were there any specific challenges or advantages in
understanding the paper (e.g.~did the authors provide sufficient
background information to understand experimental logic, were methods
explained adequately, were any specific assumptions made, were
conclusions justified based on the evidence, were the figures or tables
useful and easy to
understand)?}\label{were-there-any-specific-challenges-or-advantages-in-understanding-the-paper-e.g.did-the-authors-provide-sufficient-background-information-to-understand-experimental-logic-were-methods-explained-adequately-were-any-specific-assumptions-made-were-conclusions-justified-based-on-the-evidence-were-the-figures-or-tables-useful-and-easy-to-understand-2}

This paper was generally well described, and background information was
sufficient. The figures included were helpful and useful and aid in the
understanding. Authors cited various literature resources to back up
their arguments.

\subsection{\texorpdfstring{Problem set\_03 ``Metagenomics: Genomic
Analysis of Microbial
Communities''}{Problem set\_03 Metagenomics: Genomic Analysis of Microbial Communities}}\label{problem-set_03-metagenomics-genomic-analysis-of-microbial-communities}

\subsubsection{Learning objectives:}\label{learning-objectives-5}

Specific emphasis should be placed on the process used to find the
answer. Be as comprehensive as possible e.g.~provide URLs for web
sources, literature citations, etc.

\subsubsection{Specific Questions:}\label{specific-questions-4}

\paragraph{How many prokaryotic divisions have been described and how
many have no cultured representatives (microbial dark
matter)?}\label{how-many-prokaryotic-divisions-have-been-described-and-how-many-have-no-cultured-representatives-microbial-dark-matter}

\url{https://www.sciencedirect.com/science/article/pii/S1369527416300558}
\url{http://mbio.asm.org/content/7/3/e00201-16.full}

In 2016, 89 bacterial phyla and 20 archeal phyla have been decribed via
16s rRNA databases. However, there could be uo to 15000 bacterial phyla,
since most of the microbes are living life in ``shadow bioshphere''.

In 2003, 26 of the 52 major bacterial phyla have been cultivated
(probably more now).

\paragraph{How many metagenome sequencing projects are currently
available in the public domain and what types of environments are they
sourced
from?}\label{how-many-metagenome-sequencing-projects-are-currently-available-in-the-public-domain-and-what-types-of-environments-are-they-sourced-from}

\url{https://www.ebi.ac.uk/metagenomics/}

There are currently around 110,217 metagenome sequencing projects on EB
database which is just a small fraction of projects. They sourced from
different types of the environments such as soil, aquatics, sediments.

\paragraph{What types of on-line resources are available for warehousing
and/or analyzing environmental sequence information (provide names, URLS
and
applications)?}\label{what-types-of-on-line-resources-are-available-for-warehousing-andor-analyzing-environmental-sequence-information-provide-names-urls-and-applications}

Shot gun metagenomics:

Assembly- EULERT \url{https://omictools.com/euler-sr-tool} Binning-
S-Gcom Annotation- KEGG \url{http://www.genome.jp/kegg/annotation/}
Analysis pipelines- Megan 5
\url{http://ab.inf.uni-tuebingen.de/software/megan5/}

Marker Gene Metagenomics:

Standalone- OTUbase Analysis pipelines- SILVA Denoising- Amplicon Noise
Databases- Ribosomal Database Project (RDP)

\paragraph{What is the difference between phylogenetic and functional
gene anchors and how can they be used in metagenome
analysis?}\label{what-is-the-difference-between-phylogenetic-and-functional-gene-anchors-and-how-can-they-be-used-in-metagenome-analysis}

Phylogenetic: vertical gene transfer, carry phyogenic information,
allowing tree creconstruction, taxonomic and ideally single copy.

Functional: more horizontal gene, identify specific biogeochemical
functions associated with measurable effects, not as useful as
phylogeny.

\paragraph{What is metagenomic sequence binning? What types of
algorithmic approaches are used to produce sequence bins? What are some
risks and opportunities associated with using sequence bins for
metabolic reconstruction of uncultivated
microorganisms?}\label{what-is-metagenomic-sequence-binning-what-types-of-algorithmic-approaches-are-used-to-produce-sequence-bins-what-are-some-risks-and-opportunities-associated-with-using-sequence-bins-for-metabolic-reconstruction-of-uncultivated-microorganisms}

\url{https://www.ncbi.nlm.nih.gov/pmc/articles/PMC3504927/})
\url{http://journals.plos.org/ploscompbiol/article?id=10.1371/journal.pcbi.1000667}

Metagenomic sequence bining is the process of grouping sequences that
come from a single genome. The algorithmic approaches that are used are
aligning along data bases or based on specific characteristics such as
G/C content.The risks that are possible is the incomplete coverafe and
contamination from different phylogeny.

\paragraph{Is there an alternative to metagenomic shotgun sequencing
that can be used to access the metabolic potential of uncultivated
microorganisms? What are some risks and opportunities associated with
this
alternative?}\label{is-there-an-alternative-to-metagenomic-shotgun-sequencing-that-can-be-used-to-access-the-metabolic-potential-of-uncultivated-microorganisms-what-are-some-risks-and-opportunities-associated-with-this-alternative}

\url{http://go.galegroup.com.ezproxy.library.ubc.ca/ps/i.do?p=HRCA\&u=ubcolumbia\&id=GALE\%7CA444914243\&v=2.1\&it=r\&sid=summon}

\url{https://www.ncbi.nlm.nih.gov/pmc/articles/PMC4441954/}

Functional screens (biochemical etc.), Gene sequencing (Nanopore),
single cell sequencing, FISH probe.

\subsection{Module 2 references:}\label{module-2-references}

\begin{enumerate}
\def\labelenumi{\arabic{enumi}.}
\item
  Madsen EL. 2005. Identifying microorganisms responsible for
  ecologically significant biogeochemical processes. Nature Reviews
  Microbiology 3:439--446.
  \href{https://www.ncbi.nlm.nih.gov/pubmed/15864265}{PMID15864265}
\item
  Martinez A, Bradley AS, Waldbauer JR, Summons RE, Delong EF. 2007.
  Proteorhodopsin photosystem gene expression enables
  photophosphorylation in a heterologous host. Proceedings of the
  National Academy of Sciences 104:5590--5595.
  \href{https://www.ncbi.nlm.nih.gov/pmc/articles/PMC1838496/}{PMC1838496}
\item
  Taupp M, Mewis K, Hallam SJ. 2011. The art and design of functional
  metagenomic screens. Current Opinion in Biotechnology 22:465--472.
  \href{https://www.ncbi.nlm.nih.gov/pubmed/21440432}{PMID 21440432}
\item
  Wooley JC, Godzik A, Friedberg I. 2010. A Primer on Metagenomics.
  Proceedings of the National Academy of Sciences Computational Biology
  6: 1-16.
  (\url{http://journals.plos.org/ploscompbiol/article?id=10.1371/journal.pcbi.1000667})
\end{enumerate}

\section{Module 3:}\label{module-3}

\subsection{\texorpdfstring{Evidence Worksheet\_ 05 ``Extensive mosaic
structure''}{Evidence Worksheet\_ 05 Extensive mosaic structure}}\label{evidence-worksheet_-05-extensive-mosaic-structure}

\href{https://www.ncbi.nlm.nih.gov/pubmed/12471157}{Welch et al 2002}

\subsubsection{Part 1: Learning
objectives:}\label{part-1-learning-objectives}

-Evaluate the concept of microbial species based on environmental
surveys and cultivation studies.

-Explain the relationship between microdiversity, genomic diversity and
metabolic potential

-Comment on the forces mediating divergence and cohesion in natural
microbial communities

\paragraph{General Questions:}\label{general-questions-2}

\paragraph{What were the main questions being
asked?}\label{what-were-the-main-questions-being-asked-3}

How does the genome of the CFT073, enterohemorrhagic E. coli EDL933 and
laboratory strain MG1655 compare to each other? How does the genotype
relate to their phenotype and the environment that they live in?

\paragraph{What were the primary methodological approaches
used?}\label{what-were-the-primary-methodological-approaches-used-3}

Cloning and sequencing. Isolation, Whole-genome libraries, sequence
analysis and annotation.

\paragraph{Summarize the main results or
findings.}\label{summarize-the-main-results-or-findings.-3}

There are set of backbone E. coli genes that have a shared codon bias
that is not seen in the genes unique in each of the three genome. The
newly acquired genes are transfered through the horizontal gene transfer
are placed in the framework- result in the production of new strains
that adapt to different environments. Each type of E. coli possess
combination of island genes that confer its characteristics lifestyle or
disease-causing traits. The three E. coli strains only have 37.29 \% in
common (both ( genomic island that represent intersections and unique
parts of the island). Islands encode adaptive traits-\textgreater{}
fitness-\textgreater{}selection-\textgreater{} environment (host)
-\textgreater{} phatogenesis (whether an organism is infective or not).

\paragraph{Do new questions arise from the
results?}\label{do-new-questions-arise-from-the-results-3}

How do we define species in microbes? How easily can the genes be
transmitted horizontally?

\paragraph{Were there any specific challenges or advantages in
understanding the paper (e.g.~did the authors provide sufficient
background information to understand experimental logic, were methods
explained adequately, were any specific assumptions made, were
conclusions justified based on the evidence, were the figures or tables
useful and easy to
understand)?}\label{were-there-any-specific-challenges-or-advantages-in-understanding-the-paper-e.g.did-the-authors-provide-sufficient-background-information-to-understand-experimental-logic-were-methods-explained-adequately-were-any-specific-assumptions-made-were-conclusions-justified-based-on-the-evidence-were-the-figures-or-tables-useful-and-easy-to-understand-3}

This paper was challenging to understand. It provided many information,
but did not have a specific structure. It would also beneficial to talk
about other strains of the bacteria.

\subsubsection{Part 2: Learning
objectives:}\label{part-2-learning-objectives}

-Comment on the creative tension between gene loss, duplication and
acquisition as it relates to microbial genome evolution

-Identify common molecular signatures used to infer genomic identity and
cohesion

\begin{itemize}
\tightlist
\item
  Differentiate between mobile elements and different modes of gene
  transfer
\end{itemize}

\paragraph{Based on your reading and discussion notes, explain the
meaning and content of the following figure derived from the comparative
genomic analysis of three E. coli genomes by Welch et al.Remember that
CFT073 is a uropathogenic strain and that EDL933 is an enterohemorrhagic
strain. Explain how this study relates to your understanding of ecotype
diversity. Provide a definition of ecotype in the context of the human
body. Explain why certain subsets of genes in CFT073 provide adaptive
traits under your ecological model and speculate on their mode of
vertical descent or gene
transfer.}\label{based-on-your-reading-and-discussion-notes-explain-the-meaning-and-content-of-the-following-figure-derived-from-the-comparative-genomic-analysis-of-three-e.-coli-genomes-by-welch-et-al.remember-that-cft073-is-a-uropathogenic-strain-and-that-edl933-is-an-enterohemorrhagic-strain.-explain-how-this-study-relates-to-your-understanding-of-ecotype-diversity.-provide-a-definition-of-ecotype-in-the-context-of-the-human-body.-explain-why-certain-subsets-of-genes-in-cft073-provide-adaptive-traits-under-your-ecological-model-and-speculate-on-their-mode-of-vertical-descent-or-gene-transfer.}

The figure is the comparision between the location and sizes of CFT and
EDL. Different ecosystems applied different environmental pressures.
conditions that pressure the same species to diverge into different
strains to adapt to the environment- It is the adaption required for
survival. Ecotype definition is equivalent to that of strain. Islands
are transfered through horizontal gene transfers, but the backbone
through vertical transfer. For uropathogenic strains of E. coli, island
acquisition resulted in the capability to infect the urinary tract and
bloodstream and evade host defenses without compromising the ability to
harmlessly colonize the intestine. If they don't acquire this Horizontal
gene transfer they die.

\subsection{\texorpdfstring{Problem set\_04 ``Fine-scale phylogenetic
architecture''}{Problem set\_04 Fine-scale phylogenetic architecture}}\label{problem-set_04-fine-scale-phylogenetic-architecture}

\subsubsection{Learning objectives:}\label{learning-objectives-6}

Gain experience estimating diversity within a hypothetical microbial
community

\subsubsection{Outline:}\label{outline}

In class Day 1:

\begin{enumerate}
\def\labelenumi{\arabic{enumi}.}
\tightlist
\item
  Define and describe species within your group's ``microbial''
  community.
\item
  Count and record individuals within your defined species groups.
\item
  Remix all species together to reform the original community.
\item
  Each person in your group takes a random sample of the community
  (\emph{i.e.} devide up the candy).
\end{enumerate}

Assignment:

\begin{enumerate}
\def\labelenumi{\arabic{enumi}.}
\setcounter{enumi}{4}
\tightlist
\item
  Individually, complete a collection curve for your sample.
\item
  Calculate alpha-diversity based on your original total community and
  your individual sample.
\end{enumerate}

In class Day 2:

\begin{enumerate}
\def\labelenumi{\arabic{enumi}.}
\setcounter{enumi}{6}
\tightlist
\item
  Compare diversity between groups.
\end{enumerate}

\paragraph{Part 1: Description and
enumeration}\label{part-1-description-and-enumeration}

Obtain a collection of ``microbial'' cells from ``seawater''. The cells
were concentrated from different depth intervals by a marine
microbiologist travelling along the Line-P transect in the northeast
subarctic Pacific Ocean off the coast of Vancouver Island British
Columbia.

Sort out and identify different microbial ``species'' based on shared
properties or traits. Record your data in this Rmarkdown using the
example data as a guide.

Once you have defined your binning criteria, separate the cells using
the sampling bags provided. These operational taxonomic units (OTUs)
will be considered separate ``species''. This problem set is based on
content available at \href{http://cnx.org/content/m12179/latest/}{What
is Biodiversity}.

\begin{Shaded}
\begin{Highlighting}[]
\KeywordTok{library}\NormalTok{(kableExtra)}
\KeywordTok{library}\NormalTok{(knitr)}
\KeywordTok{library}\NormalTok{(tidyverse)}
\end{Highlighting}
\end{Shaded}

\begin{verbatim}
## -- Attaching packages --------------------------------------- tidyverse 1.2.1 --
\end{verbatim}

\begin{verbatim}
## √ ggplot2 2.2.1     √ purrr   0.2.4
## √ tibble  1.4.2     √ dplyr   0.7.4
## √ tidyr   0.8.0     √ stringr 1.3.0
## √ readr   1.1.1     √ forcats 0.3.0
\end{verbatim}

\begin{verbatim}
## -- Conflicts ------------------------------------------ tidyverse_conflicts() --
## x dplyr::filter() masks stats::filter()
## x dplyr::lag()    masks stats::lag()
\end{verbatim}

\begin{Shaded}
\begin{Highlighting}[]
\KeywordTok{library}\NormalTok{(dplyr)}
\end{Highlighting}
\end{Shaded}

For your community:

Construct a table listing each species, its distinguishing
characteristics, the name you have given it, and the number of
occurrences of the species in the collection. Ask yourself if your
collection of microbial cells from seawater represents the actual
diversity of microorganisms inhabiting waters along the Line-P transect.
Were the majority of different species sampled or were many missed?

\begin{Shaded}
\begin{Highlighting}[]
\NormalTok{example_data4 =}\StringTok{ }\KeywordTok{data.frame}\NormalTok{(}
  \DataTypeTok{number =} \KeywordTok{c}\NormalTok{(}\DecValTok{1}\NormalTok{,}\DecValTok{2}\NormalTok{,}\DecValTok{3}\NormalTok{,}\DecValTok{4}\NormalTok{,}\DecValTok{5}\NormalTok{,}\DecValTok{6}\NormalTok{,}\DecValTok{7}\NormalTok{,}\DecValTok{8}\NormalTok{,}\DecValTok{9}\NormalTok{,}\DecValTok{10}\NormalTok{,}\DecValTok{11}\NormalTok{,}\DecValTok{12}\NormalTok{,}\DecValTok{13}\NormalTok{,}\DecValTok{14}\NormalTok{),}
  \DataTypeTok{name =} \KeywordTok{c}\NormalTok{(}\StringTok{"vines"}\NormalTok{, }\StringTok{"bricks"}\NormalTok{, }\StringTok{"skittles"}\NormalTok{, }\StringTok{"mikes & ikes"}\NormalTok{, }\StringTok{"gummy bears"}\NormalTok{, }\StringTok{"m&ms"}\NormalTok{, }\StringTok{"hershey kisses"}\NormalTok{, }\StringTok{"sour bear"}\NormalTok{, }\StringTok{"sour fruit"}\NormalTok{, }\StringTok{"sour hexa"}\NormalTok{, }\StringTok{"sour bottle"}\NormalTok{, }\StringTok{"sour swirl"}\NormalTok{, }\StringTok{"jubes"}\NormalTok{, }\StringTok{"wine candy"}\NormalTok{),}
  \DataTypeTok{Characteristics=} \KeywordTok{c}\NormalTok{(}\StringTok{"wine shaped"}\NormalTok{, }\StringTok{"swirl candy"}\NormalTok{, }\StringTok{"silver colored"}\NormalTok{, }\StringTok{"red licorice"}\NormalTok{, }\StringTok{"octopus shaped"}\NormalTok{, }\StringTok{"fruit-flavoured sweets"}\NormalTok{, }\StringTok{"fruit shaped and sour"}\NormalTok{, }\StringTok{"elongated chewy candies"}\NormalTok{, }\StringTok{"colored candies"}\NormalTok{, }\StringTok{"button-shaped candy"}\NormalTok{, }\StringTok{"brick shaped"}\NormalTok{, }\StringTok{"bottle shaped"}\NormalTok{, }\StringTok{"bear shaped"}\NormalTok{, }\StringTok{"bear and sour"}\NormalTok{),}
  \DataTypeTok{Occurances =} \KeywordTok{c}\NormalTok{(}\DecValTok{14}\NormalTok{,}\DecValTok{18}\NormalTok{,}\DecValTok{187}\NormalTok{,}\DecValTok{174}\NormalTok{,}\DecValTok{101}\NormalTok{,}\DecValTok{241}\NormalTok{,}\DecValTok{16}\NormalTok{,}\DecValTok{3}\NormalTok{,}\DecValTok{2}\NormalTok{,}\DecValTok{6}\NormalTok{,}\DecValTok{3}\NormalTok{,}\DecValTok{3}\NormalTok{,}\DecValTok{24}\NormalTok{,}\DecValTok{9}\NormalTok{)}
\NormalTok{)}
\end{Highlighting}
\end{Shaded}

\begin{Shaded}
\begin{Highlighting}[]
\NormalTok{example_data4 }\OperatorTok\StringTok{ }
\StringTok{  }\KeywordTok{kable}\NormalTok{(}\StringTok{"html"}\NormalTok{) }\OperatorTok
\StringTok{  }\KeywordTok{kable_styling}\NormalTok{(}\DataTypeTok{bootstrap_options =} \StringTok{"striped"}\NormalTok{, }\DataTypeTok{font_size =} \DecValTok{10}\NormalTok{, }\DataTypeTok{full_width =}\NormalTok{ F)}
\end{Highlighting}
\end{Shaded}

number

name

Characteristics

Occurances

1

vines

wine shaped

14

2

bricks

swirl candy

18

3

skittles

silver colored

187

4

mikes \& ikes

red licorice

174

5

gummy bears

octopus shaped

101

6

m\&ms

fruit-flavoured sweets

241

7

hershey kisses

fruit shaped and sour

16

8

sour bear

elongated chewy candies

3

9

sour fruit

colored candies

2

10

sour hexa

button-shaped candy

6

11

sour bottle

brick shaped

3

12

sour swirl

bottle shaped

3

13

jubes

bear shaped

24

14

wine candy

bear and sour

9

Ask yourself if your collection of microbial cells from seawater
represents the actual diversity of microorganisms inhabiting waters
along the Line-P transect. Were the majority of different species
sampled or were many missed? As it can be observed in the graph, the
majority of the microorganisms (candies in this case) were sampled and
only 2 have been missed. In order to explain whether or not the
collection (sample) represents the actual diversity of microorganisms,
collector's curve is constructed.

\paragraph{Part 2: Collector's curve}\label{part-2-collectors-curve}

To help answer the questions raised in Part 1, you will conduct a simple
but informative analysis that is a standard practice in biodiversity
surveys. This analysis involves constructing a collector's curve that
plots the cumulative number of species observed along the y-axis and the
cumulative number of individuals classified along the x-axis. This curve
is an increasing function with a slope that will decrease as more
individuals are classified and as fewer species remain to be identified.
If sampling stops while the curve is still rapidly increasing then this
indicates that sampling is incomplete and many species remain
undetected. Alternatively, if the slope of the curve reaches zero
(flattens out), sampling is likely more than adequate.

To construct the curve for your samples, choose a cell within the
collection at random. This will be your first data point, such that X =
1 and Y = 1. Next, move consistently in any direction to a new cell and
record whether it is different from the first. In this step X = 2, but Y
may remain 1 or change to 2 if the individual represents a new species.
Repeat this process until you have proceeded through all cells in your
collection.

For example, we load in these data.

\begin{Shaded}
\begin{Highlighting}[]
\NormalTok{example_data2 =}\StringTok{ }\KeywordTok{data.frame}\NormalTok{(}
  \DataTypeTok{x =} \KeywordTok{c}\NormalTok{(}\DecValTok{1}\NormalTok{,}\DecValTok{2}\NormalTok{,}\DecValTok{3}\NormalTok{,}\DecValTok{4}\NormalTok{,}\DecValTok{5}\NormalTok{,}\DecValTok{6}\NormalTok{,}\DecValTok{7}\NormalTok{,}\DecValTok{8}\NormalTok{,}\DecValTok{9}\NormalTok{,}\DecValTok{10}\NormalTok{),}
  \DataTypeTok{y =} \KeywordTok{c}\NormalTok{(}\DecValTok{1}\NormalTok{,}\DecValTok{2}\NormalTok{,}\DecValTok{3}\NormalTok{,}\DecValTok{4}\NormalTok{,}\DecValTok{4}\NormalTok{,}\DecValTok{5}\NormalTok{,}\DecValTok{5}\NormalTok{,}\DecValTok{5}\NormalTok{,}\DecValTok{6}\NormalTok{,}\DecValTok{6}\NormalTok{)}
\NormalTok{)}
\end{Highlighting}
\end{Shaded}

And then create a plot. We will use a scatterplot (geom\_point) to plot
the raw data and then add a smoother to see the overall trend of the
data.

\begin{Shaded}
\begin{Highlighting}[]
\KeywordTok{ggplot}\NormalTok{(example_data2, }\KeywordTok{aes}\NormalTok{(}\DataTypeTok{x=}\NormalTok{x, }\DataTypeTok{y=}\NormalTok{y)) }\OperatorTok{+}
\StringTok{  }\KeywordTok{geom_point}\NormalTok{() }\OperatorTok{+}
\StringTok{  }\KeywordTok{geom_smooth}\NormalTok{() }\OperatorTok{+}
\StringTok{  }\KeywordTok{labs}\NormalTok{(}\DataTypeTok{x=}\StringTok{"Cumulative number of individuals classified"}\NormalTok{, }\DataTypeTok{y=}\StringTok{"Cumulative number of species observed"}\NormalTok{)}
\end{Highlighting}
\end{Shaded}

\begin{verbatim}
## `geom_smooth()` using method = 'loess'
\end{verbatim}

\includegraphics{33305137_MICB425_Portfolio_files/figure-latex/unnamed-chunk-7-1.pdf}

For your sample:

\begin{itemize}
\tightlist
\item
  Create a collector's curve for your sample (not the entire original
  community).
\item
  Does the curve flatten out? If so, after how many individual cells
  have been collected? The curve starts flatting out after 100
  individuals.
\item
  What can you conclude from the shape of your collector's curve as to
  your depth of sampling? From the shape of the collector's curve, it
  can be observed that the slope decreases as more individueals are
  classified. Additionally, the sampling doesn't stop while the curve is
  increasing so the sampling is not incomplete. Another point to mention
  is that the curve reached zero at the end, showing that sampling was
  adequate.
\end{itemize}

\begin{Shaded}
\begin{Highlighting}[]
\NormalTok{example_data4=}\StringTok{ }\KeywordTok{data.frame}\NormalTok{(}
  \DataTypeTok{x=} \KeywordTok{c}\NormalTok{(}\DecValTok{1}\OperatorTok{:}\DecValTok{178}\NormalTok{), }
  \DataTypeTok{y=}\KeywordTok{c}\NormalTok{(}\DecValTok{1}\NormalTok{,}\DecValTok{1}\NormalTok{,}\DecValTok{2}\NormalTok{,}\DecValTok{2}\NormalTok{,}\DecValTok{2}\NormalTok{,}\DecValTok{3}\NormalTok{,}\DecValTok{3}\NormalTok{,}\DecValTok{3}\NormalTok{,}\DecValTok{4}\NormalTok{,}\DecValTok{4}\NormalTok{,}\DecValTok{4}\NormalTok{,}\DecValTok{4}\NormalTok{,}\DecValTok{5}\NormalTok{,}\DecValTok{5}\NormalTok{,}\DecValTok{5}\NormalTok{,}\DecValTok{5}\NormalTok{,}\DecValTok{5}\NormalTok{,}\DecValTok{5}\NormalTok{,}\DecValTok{6}\NormalTok{,}\DecValTok{6}\NormalTok{,}\DecValTok{6}\NormalTok{,}\DecValTok{6}\NormalTok{,}\DecValTok{6}\NormalTok{,}\DecValTok{6}\NormalTok{,}\DecValTok{7}\NormalTok{,}\DecValTok{7}\NormalTok{,}\DecValTok{7}\NormalTok{,}\DecValTok{7}\NormalTok{,}\DecValTok{7}\NormalTok{,}\DecValTok{7}\NormalTok{,}\DecValTok{7}\NormalTok{,}\DecValTok{7}\NormalTok{,}\DecValTok{7}\NormalTok{,}\DecValTok{7}\NormalTok{,}\DecValTok{7}\NormalTok{,}\DecValTok{8}\NormalTok{,}\DecValTok{8}\NormalTok{,}\DecValTok{8}\NormalTok{,}\DecValTok{8}\NormalTok{,}\DecValTok{8}\NormalTok{,}\DecValTok{8}\NormalTok{,}\DecValTok{8}\NormalTok{,}\DecValTok{8}\NormalTok{,}\DecValTok{8}\NormalTok{,}\DecValTok{8}\NormalTok{,}\DecValTok{8}\NormalTok{,}\DecValTok{8}\NormalTok{,}\DecValTok{8}\NormalTok{,}\DecValTok{8}\NormalTok{,}\DecValTok{8}\NormalTok{,}\DecValTok{9}\NormalTok{,}\DecValTok{9}\NormalTok{,}\DecValTok{9}\NormalTok{,}\DecValTok{9}\NormalTok{,}\DecValTok{9}\NormalTok{,}\DecValTok{9}\NormalTok{,}\DecValTok{9}\NormalTok{,}\DecValTok{9}\NormalTok{,}\DecValTok{9}\NormalTok{,}\DecValTok{9}\NormalTok{,}\DecValTok{9}\NormalTok{,}\DecValTok{9}\NormalTok{,}\DecValTok{9}\NormalTok{,}\DecValTok{9}\NormalTok{,}\DecValTok{9}\NormalTok{,}\DecValTok{9}\NormalTok{,}\DecValTok{10}\NormalTok{,}\DecValTok{10}\NormalTok{,}\DecValTok{10}\NormalTok{,}\DecValTok{10}\NormalTok{,}\DecValTok{10}\NormalTok{,}\DecValTok{10}\NormalTok{,}\DecValTok{10}\NormalTok{,}\DecValTok{10}\NormalTok{,}\DecValTok{10}\NormalTok{,}\DecValTok{10}\NormalTok{,}\DecValTok{10}\NormalTok{,}\DecValTok{10}\NormalTok{,}\DecValTok{10}\NormalTok{,}\DecValTok{10}\NormalTok{,}\DecValTok{11}\NormalTok{,}\DecValTok{11}\NormalTok{,}\DecValTok{11}\NormalTok{,}\DecValTok{11}\NormalTok{,}\DecValTok{11}\NormalTok{,}\DecValTok{11}\NormalTok{,}\DecValTok{11}\NormalTok{,}\DecValTok{11}\NormalTok{,}\DecValTok{11}\NormalTok{,}\DecValTok{11}\NormalTok{,}\DecValTok{11}\NormalTok{,}\DecValTok{11}\NormalTok{,}\DecValTok{11}\NormalTok{,}\DecValTok{11}\NormalTok{,}\DecValTok{11}\NormalTok{,}\DecValTok{11}\NormalTok{,}\DecValTok{12}\NormalTok{,}\DecValTok{12}\NormalTok{,}\DecValTok{12}\NormalTok{,}\DecValTok{12}\NormalTok{,}\DecValTok{12}\NormalTok{,}\DecValTok{12}\NormalTok{,}\DecValTok{12}\NormalTok{,}\DecValTok{12}\NormalTok{,}\DecValTok{12}\NormalTok{,}\DecValTok{12}\NormalTok{,}\DecValTok{12}\NormalTok{,}\DecValTok{12}\NormalTok{,}\DecValTok{12}\NormalTok{,}\DecValTok{12}\NormalTok{,}\DecValTok{12}\NormalTok{,}\DecValTok{12}\NormalTok{,}\DecValTok{12}\NormalTok{,}\DecValTok{12}\NormalTok{,}\DecValTok{12}\NormalTok{,}\DecValTok{12}\NormalTok{,}\DecValTok{12}\NormalTok{,}\DecValTok{12}\NormalTok{,}\DecValTok{12}\NormalTok{,}\DecValTok{12}\NormalTok{,}\DecValTok{13}\NormalTok{,}\DecValTok{13}\NormalTok{,}\DecValTok{13}\NormalTok{,}\DecValTok{13}\NormalTok{,}\DecValTok{13}\NormalTok{,}\DecValTok{13}\NormalTok{,}\DecValTok{13}\NormalTok{,}\DecValTok{13}\NormalTok{,}\DecValTok{13}\NormalTok{,}\DecValTok{13}\NormalTok{,}\DecValTok{13}\NormalTok{,}\DecValTok{13}\NormalTok{,}\DecValTok{13}\NormalTok{,}\DecValTok{13}\NormalTok{,}\DecValTok{13}\NormalTok{,}\DecValTok{13}\NormalTok{,}\DecValTok{13}\NormalTok{,}\DecValTok{13}\NormalTok{,}\DecValTok{13}\NormalTok{,}\DecValTok{13}\NormalTok{,}\DecValTok{13}\NormalTok{,}\DecValTok{13}\NormalTok{,}\DecValTok{13}\NormalTok{,}\DecValTok{13}\NormalTok{,}\DecValTok{13}\NormalTok{,}\DecValTok{13}\NormalTok{,}\DecValTok{13}\NormalTok{,}\DecValTok{13}\NormalTok{,}\DecValTok{14}\NormalTok{,}\DecValTok{14}\NormalTok{,}\DecValTok{14}\NormalTok{,}\DecValTok{14}\NormalTok{,}\DecValTok{14}\NormalTok{,}\DecValTok{14}\NormalTok{,}\DecValTok{14}\NormalTok{,}\DecValTok{14}\NormalTok{,}\DecValTok{14}\NormalTok{,}\DecValTok{14}\NormalTok{,}\DecValTok{14}\NormalTok{,}\DecValTok{14}\NormalTok{,}\DecValTok{14}\NormalTok{,}\DecValTok{14}\NormalTok{,}\DecValTok{14}\NormalTok{,}\DecValTok{14}\NormalTok{,}\DecValTok{14}\NormalTok{,}\DecValTok{14}\NormalTok{,}\DecValTok{14}\NormalTok{,}\DecValTok{14}\NormalTok{,}\DecValTok{14}\NormalTok{,}\DecValTok{14}\NormalTok{,}\DecValTok{14}\NormalTok{,}\DecValTok{14}\NormalTok{,}\DecValTok{14}\NormalTok{,}\DecValTok{14}\NormalTok{,}\DecValTok{14}\NormalTok{,}\DecValTok{14}\NormalTok{,}\DecValTok{14}\NormalTok{,}\DecValTok{14}\NormalTok{)}
\NormalTok{)}
\end{Highlighting}
\end{Shaded}

\begin{Shaded}
\begin{Highlighting}[]
\KeywordTok{ggplot}\NormalTok{(example_data4, }\KeywordTok{aes}\NormalTok{(}\DataTypeTok{x=}\NormalTok{x, }\DataTypeTok{y=}\NormalTok{y)) }\OperatorTok{+}
\StringTok{  }\KeywordTok{geom_point}\NormalTok{() }\OperatorTok{+}
\StringTok{  }\KeywordTok{geom_smooth}\NormalTok{() }\OperatorTok{+}
\StringTok{  }\KeywordTok{labs}\NormalTok{(}\DataTypeTok{x=}\StringTok{"Cumulative number of individuals classified"}\NormalTok{, }\DataTypeTok{y=}\StringTok{"Cumulative number of species observed"}\NormalTok{)}
\end{Highlighting}
\end{Shaded}

\begin{verbatim}
## `geom_smooth()` using method = 'loess'
\end{verbatim}

\includegraphics{33305137_MICB425_Portfolio_files/figure-latex/unnamed-chunk-9-1.pdf}

\paragraph{Part 3: Diversity estimates (alpha
diversity)}\label{part-3-diversity-estimates-alpha-diversity}

Using the table from Part 1, calculate species diversity using the
following indices or metrics.

\subparagraph{Diversity: Simpson Reciprocal
Index}\label{diversity-simpson-reciprocal-index}

\(\frac{1}{D}\) where \(D = \sum p_i^2\)

\(p_i\) = the fractional abundance of the \(i^{th}\) species

For example, using the example data 1 with 3 species with 2, 4, and 1
individuals each, D =

\begin{Shaded}
\begin{Highlighting}[]
\NormalTok{species1 =}\StringTok{ }\DecValTok{2}\OperatorTok{/}\NormalTok{(}\DecValTok{2}\OperatorTok{+}\DecValTok{4}\OperatorTok{+}\DecValTok{1}\NormalTok{)}
\NormalTok{species2 =}\StringTok{ }\DecValTok{4}\OperatorTok{/}\NormalTok{(}\DecValTok{2}\OperatorTok{+}\DecValTok{4}\OperatorTok{+}\DecValTok{1}\NormalTok{)}
\NormalTok{species3 =}\StringTok{ }\DecValTok{1}\OperatorTok{/}\NormalTok{(}\DecValTok{2}\OperatorTok{+}\DecValTok{4}\OperatorTok{+}\DecValTok{1}\NormalTok{)}

\DecValTok{1} \OperatorTok{/}\StringTok{ }\NormalTok{(species1}\OperatorTok{^}\DecValTok{2} \OperatorTok{+}\StringTok{ }\NormalTok{species2}\OperatorTok{^}\DecValTok{2} \OperatorTok{+}\StringTok{ }\NormalTok{species3}\OperatorTok{^}\DecValTok{2}\NormalTok{)}
\end{Highlighting}
\end{Shaded}

\begin{verbatim}
## [1] 2.333333
\end{verbatim}

The higher the value is, the greater the diversity. The maximum value is
the number of species in the sample, which occurs when all species
contain an equal number of individuals. Because the index reflects the
number of species present (richness) and the relative proportions of
each species with a community (evenness), this metric is a diveristy
metric. Consider that a community can have the same number of species
(equal richness) but manifest a skewed distribution in the proportion of
each species (unequal evenness), which would result in different
diveristy values.

\begin{itemize}
\tightlist
\item
  What is the Simpson Reciprocal Index for your sample? 4.75164967
\item
  What is the Simpson Reciprocal Index for your original total
  community? 4.75164967
\end{itemize}

\subparagraph{Richness: Chao1 richness
estimator}\label{richness-chao1-richness-estimator}

Another way to calculate diversity is to estimate the number of species
that are present in a sample based on the empirical data to give an
upper boundary of the richness of a sample. Here, we use the Chao1
richness estimator.

\(S_{chao1} = S_{obs} + \frac{a^2}{2b})\)

\(S_{obs}\) = total number of species observed a = species observed once
b = species observed twice or more

So for our previous example community of 3 species with 2, 4, and 1
individuals each, \(S_{chao1}\) =

\begin{Shaded}
\begin{Highlighting}[]
\DecValTok{3} \OperatorTok{+}\StringTok{ }\DecValTok{1}\OperatorTok{^}\DecValTok{2}\OperatorTok{/}\NormalTok{(}\DecValTok{2}\OperatorTok{*}\DecValTok{2}\NormalTok{)}
\end{Highlighting}
\end{Shaded}

\begin{verbatim}
## [1] 3.25
\end{verbatim}

\begin{itemize}
\tightlist
\item
  What is the chao1 estimate for your sample? 13.1818
\item
  What is the chao1 estimate for your original total community? 14
\end{itemize}

\paragraph{Part 4: Alpha-diversity functions in
R}\label{part-4-alpha-diversity-functions-in-r}

We've been doing the above calculations by hand, which is a very good
exercise to aid in understanding the math behind these estimates. Not
surprisingly, these same calculations can be done with R functions.
Since we just have a species table, we will use the \texttt{vegan}
package. You will need to install this package if you have not done so
previously.

\begin{Shaded}
\begin{Highlighting}[]
\KeywordTok{library}\NormalTok{(vegan)}
\end{Highlighting}
\end{Shaded}

First, we must remove the unnecesary data columns and transpose the data
so that \texttt{vegan} reads it as a species table with species as
columns and rows as samples (of which you only have 1).

\begin{Shaded}
\begin{Highlighting}[]
\NormalTok{example_data1_diversity =}\StringTok{ }
\StringTok{  }\NormalTok{example_data1 }\OperatorTok\StringTok{ }
\StringTok{  }\KeywordTok{select}\NormalTok{(name, occurences) }\OperatorTok\StringTok{ }
\StringTok{  }\KeywordTok{spread}\NormalTok{(name, occurences)}
\end{Highlighting}
\end{Shaded}

\begin{verbatim}
## Error in eval(lhs, parent, parent): object 'example_data1' not found
\end{verbatim}

\begin{Shaded}
\begin{Highlighting}[]
\NormalTok{example_data1_diversity}
\end{Highlighting}
\end{Shaded}

\begin{verbatim}
## Error in eval(expr, envir, enclos): object 'example_data1_diversity' not found
\end{verbatim}

Then we can calculate the Simpson Reciprocal Index using the
\texttt{diversity} function.

\begin{Shaded}
\begin{Highlighting}[]
\KeywordTok{diversity}\NormalTok{(example_data1_diversity, }\DataTypeTok{index=}\StringTok{"invsimpson"}\NormalTok{)}
\end{Highlighting}
\end{Shaded}

\begin{verbatim}
## Error in as.matrix(x): object 'example_data1_diversity' not found
\end{verbatim}

And we can calculate the Chao1 richness estimator (and others by
default) with the the \texttt{specpool} function for extrapolated
species richness. This function rounds to the nearest whole number so
the value will be slightly different that what you've calculated above.

\begin{Shaded}
\begin{Highlighting}[]
\KeywordTok{specpool}\NormalTok{(example_data1_diversity)}
\end{Highlighting}
\end{Shaded}

\begin{verbatim}
## Error in as.matrix(x): object 'example_data1_diversity' not found
\end{verbatim}

In Project 1, you will also see functions for calculating
alpha-diversity in the \texttt{phyloseq} package since we will be
working with data in that form.

For your sample:

\begin{itemize}
\tightlist
\item
  What are the Simpson Reciprocal Indices for your sample and community
  using the R function? 4.855468 for the sample
\item
  What are the chao1 estimates for your sample and community using the R
  function? 12 for the sample

  \begin{itemize}
  \tightlist
  \item
    Verify that these values match your previous calculations. The
    values obtain using R were fairly similar to those calculated by
    spread sheet.
  \end{itemize}
\end{itemize}

\begin{Shaded}
\begin{Highlighting}[]
\NormalTok{example_data3_diversity =}\StringTok{ }
\StringTok{  }\NormalTok{example_data3 }\OperatorTok\StringTok{ }
\StringTok{  }\KeywordTok{select}\NormalTok{(name, occurences) }\OperatorTok\StringTok{ }
\StringTok{  }\KeywordTok{spread}\NormalTok{(name, occurences)}
\end{Highlighting}
\end{Shaded}

\begin{verbatim}
## Error in eval(lhs, parent, parent): object 'example_data3' not found
\end{verbatim}

\begin{Shaded}
\begin{Highlighting}[]
\NormalTok{example_data3_diversity}
\end{Highlighting}
\end{Shaded}

\begin{verbatim}
## Error in eval(expr, envir, enclos): object 'example_data3_diversity' not found
\end{verbatim}

\begin{Shaded}
\begin{Highlighting}[]
\KeywordTok{diversity}\NormalTok{(example_data3_diversity, }\DataTypeTok{index=}\StringTok{"invsimpson"}\NormalTok{)}
\end{Highlighting}
\end{Shaded}

\begin{verbatim}
## Error in as.matrix(x): object 'example_data3_diversity' not found
\end{verbatim}

\begin{Shaded}
\begin{Highlighting}[]
\KeywordTok{specpool}\NormalTok{(example_data3_diversity)}
\end{Highlighting}
\end{Shaded}

\begin{verbatim}
## Error in as.matrix(x): object 'example_data3_diversity' not found
\end{verbatim}

\paragraph{Part 5: Concluding
activity}\label{part-5-concluding-activity}

If you are stuck on some of these final questions, reading the
\href{https://www.ncbi.nlm.nih.gov/pubmed/19725865}{Kunin et al. 2010}
and \href{https://www.ncbi.nlm.nih.gov/pubmed/23760801}{Lundin et al.
2012} papers may provide helpful insights.

\begin{itemize}
\tightlist
\item
  How does the measure of diversity depend on the definition of species
  in your samples?
\end{itemize}

Measure of diversity depends heavily on our definitions of species.
Based on the definitions of species, we bin the candies.

\begin{itemize}
\tightlist
\item
  Can you think of alternative ways to cluster or bin your data that
  might change the observed number of species?
\end{itemize}

Yes, another way was to define species by the type of the candy and also
the color of the candies.

\begin{itemize}
\tightlist
\item
  How might different sequencing technologies influence observed
  diversity in a sample?
\end{itemize}

Different sequencing technology can define a species differenly. This
would put a challenge on the researchers who use different technoloies
and might not be able to compare their data in future studies.

\subsection{Module 3 references:}\label{module-3-references}

\begin{enumerate}
\def\labelenumi{\arabic{enumi}.}
\item
  Callahan BJ, Mcmurdie PJ, Holmes SP. 2017. Exact sequence variants
  should replace operational taxonomic units in marker gene data
  analysis. The ISME Journal 11: 2639-2643.
  \href{https://www.ncbi.nlm.nih.gov/pubmed/28731476}{PMC5702726}
\item
  Gaudet AD, Ramer LM, Nakonechny J, Cragg JJ, Ramer MS. 2010.
  Small-Group Learning in an Upper-Level University Biology Class
  Enhances Academic Performance and Student Attitudes Toward Group Work.
  PLoS ONE 5: 1-10.
  (\url{http://journals.plos.org/plosone/article?id=10.1371/journal.pone.0015821})
\item
  Hallam SJ, Torres-Beltrán M, Hawley AK. 2017. Monitoring microbial
  responses to ocean deoxygenation in a model oxygen minimum zone.
  Scientific Data 4:170158.
  \href{https://www.ncbi.nlm.nih.gov/pmc/articles/PMC5663219/}{PMC5663219}
\item
  Hawley AK, Torres-Beltrán M, Zaikova E, Walsh DA, Mueller A, Scofield
  M, Kheirandish S, Payne C, Pakhomova L, Bhatia M, Shevchuk O, Gies EA,
  Fairley D, Malfatti SA, Norbeck AD, Brewer HM, Pasa-Tolic L, Rio TGD,
  Suttle CA, Tringe S, Hallam SJ. 2017. A compendium of multi-omic
  sequence information from the Saanich Inlet water column. Scientific
  Data 4:170160.
  \href{https://www.ncbi.nlm.nih.gov/pubmed/29087368}{PMC5663217}
\item
  Kunin V, Engelbrektson A, Ochman H, Hugenholtz P. 2010. Wrinkles in
  the rare biosphere: pyrosequencing errors can lead to artificial
  inflation of diversity estimates. Environmental Microbiology
  12:118--123.
  \href{https://www.ncbi.nlm.nih.gov/pubmed/19725865}{PMID19725865}
\item
  Cordero OX, Ventouras L-A, Delong EF, Polz MF. 2012. Public good
  dynamics drive evolution of iron acquisition strategies in natural
  bacterioplankton populations. Proceedings of the National Academy of
  Sciences 109:20059--20064.
  \href{https://www.ncbi.nlm.nih.gov/pubmed/23169633}{PMC3523850}
\item
  Giovannoni SJ. 2012. Vitamins in the sea. Proceedings of the National
  Academy of Sciences 109:13888--18889.
  \href{https://www.ncbi.nlm.nih.gov/pubmed/22891350}{PMC3435215}
\item
  Lundin D, Severin I, Logue JB, Östman Ö, Andersson AF, Lindström ES.
  2012. Which sequencing depth is sufficient to describe patterns in
  bacterial α- and β-diversity? Environmental Microbiology Reports
  4:367--372.
  (\url{https://onlinelibrary.wiley.com/doi/abs/10.1111/j.1758-2229.2012.00345.x})
\item
  Morris JJ, Lenski RE, Zinser ER. 2012. The Black Queen Hypothesis:
  Evolution of Dependencies through Adaptive Gene Loss. mBio
  3:e00036-12.
  \href{https://www.ncbi.nlm.nih.gov/pmc/articles/PMC3315703/}{PMC3315703}
\item
  Thompson JR, Pacocha S, Pharino C, Klepac-Ceraj V, Hunt DE, Benoit J,
  Sarma-Rupavtarm R, Distel DL, Polz MF. 2005. Genotypic Diversity
  Within a Natural Coastal Bacterioplankton Population. Science
  307:1311--1313.
  \href{https://www.ncbi.nlm.nih.gov/pubmed/15731455}{PMID15731455}
\item
  Sogin ML, Morrison HG, Huber JA, Welch DM, Huse SM, Nael PR, Arrieta
  JM, Herndl GJ. 2006. Microbial diversity in the deep sea and the
  underexplored ``rare biosphere.'' Proceedings of the National Academy
  of Sciences 103:12115--12120.
  \href{https://www.ncbi.nlm.nih.gov/pubmed/16880384}{PMC1524930}
\item
  Torres-Beltrán M, Hawley AK, Capelle D, Zaikova E, Walsh DA, Mueller
  A, Scofield M, Payne C, Pakhomova L, Kheirandish S, Finke J, Bhatia M,
  Shevchuk O, Gies EA, Fairley D, Michiels C, Suttle CA, Whitney F,
  Crowe SA, Tortell PD, Hallam SJ. 2017. A compendium of geochemical
  information from the Saanich Inlet water column. Scientific Data
  4:170159.
  \href{https://www.ncbi.nlm.nih.gov/pubmed/29087371}{PMC5663218}
\item
  Welch RA, Burland V, Plunkett G, Redford P, Roesch P, Rasko D, Buckles
  EL, Liou SR, Boutin A, Hackett J, Stroud D, Mayhew GF, Rose DJ, Zhou
  S, Schwartz DC, Perna NT, Mobley HLT, Donnenberg MS, Blattner FR.
  2002. Extensive mosaic structure revealed by the complete genome
  sequence of uropathogenic Escherichia coli. Proceedings of the
  National Academy of Sciences 99:17020--17024.
  \href{https://www.ncbi.nlm.nih.gov/pubmed/12471157}{PMC139262}
\end{enumerate}


\end{document}
